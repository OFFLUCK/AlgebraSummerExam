\documentclass[11pt,a4paper]{article}
\usepackage[utf8]{inputenc}
\usepackage[russian]{babel}
\usepackage[OT1]{fontenc}
\usepackage{amsmath}
\usepackage{hyperref}
\usepackage{amsfonts}
\usepackage{amssymb}
\usepackage{mathtools}
\usepackage[left=2cm,right=2cm,top=2cm,bottom=2cm]{geometry}
\usepackage[italicdiff]{physics}

\hypersetup{
    colorlinks=true,
    linkcolor=blue,
    filecolor=magenta,      
    urlcolor=blue,
    pdftitle={Overleaf Example},
    pdfpagemode=FullScreen,
    }

\newcommand{\proof}{$\square$ }
\newcommand{\qed}{\hfill$\blacksquare$}
\newcommand{\pd}{\partialderivative}
\newcommand{\p}{\partial}

\newcommand{\R}{\mathbb{R}}

\setlength{\parskip}{1em}
\setlength{\parindent}{0pt}

\usepackage{fancyhdr}
\pagestyle{fancy}
\fancyhf{}
\fancyhead[C]{\textsf{ by \href{https://youtu.be/dQw4w9WgXcQ}{Oleh\&Vanyok} }}

\begin{document}

\begin{center}

\begin{huge}
\textsf{Линейная Алгебра\\1 курс}
\end{huge}

\vspace{5mm}

\begin{LARGE}
\textsf{\textbf{Теория для экзамена 4 модуля}}
\end{LARGE}

\end{center}

\textbf{1. Дайте определение линейного функционала.\\}

\textbf{2. Дайте определение сопряженного пространства.\\}

\textbf{3. Выпишите формулу для преобразования координат ковектора при переходе к другому базису.\\}

\textbf{4. Дайте определение взаимных базисов.\\}

\textbf{5. Дайте определение биортогонального базиса.\\}

\textbf{6. Дайте определение сопряженного оператора в произвольном (не обязательно евклидовом) пространстве.\\}

\textbf{7. Сформулируйте определение алгебры над полем. Приведите два примера.\\}

\textbf{8. Сформулируйте определение тензора. Приведите два примера.\\}

\textbf{9. Дайте определение эллипса как геометрического места точек. Выпишите его каноническое уравнение. Что такое эксцентриситет эллипса? В каких пределах он может меняться?\\}

\textbf{10. Дайте определение гиперболы как геометрического места точек. Выпишите её каноническое уравнение. Что такое эксцентриситет гиперболы? В каких пределах он может меняться?\\}

\textbf{11. Дайте определение параболы как геометрического места точек. Выпишите её каноническое уравнение.\\}

\textbf{12. Сформулируйте теорему о классификации кривых второго порядка.\\}

\textbf{13. Дайте определение цилиндрической поверхности.\\}

\textbf{14. Дайте определение линейчатой поверхности. Приведите три примера.\\}

\textbf{15. Запишите канонические уравнения эллиптического, гиперболического и параболического цилиндров.\\}

\textbf{16. Запишите канонические уравнения эллипсоида, однополостного гиперболоида, двуполостного гиперболоида.\\}

\textbf{17. Запишите канонические уравнения эллиптического параболоида, гиперболического параболоида.\\}

\end{document}