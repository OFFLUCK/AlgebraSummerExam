\documentclass[11pt,a4paper]{article}
\usepackage[utf8]{inputenc}
\usepackage[russian]{babel}
\usepackage[OT1]{fontenc}
\usepackage{amsmath}
\usepackage{hyperref}
\usepackage{amsfonts}
\usepackage{amssymb}
\usepackage{mathtools}
\usepackage[left=2cm,right=2cm,top=2cm,bottom=2cm]{geometry}
\usepackage[italicdiff]{physics}

\hypersetup
    {
    colorlinks=true,
    linkcolor=blue,
    filecolor=magenta,      
    urlcolor=blue,
    pdftitle={Overleaf Example},
    pdfpagemode=FullScreen,
    }

\newcommand{\R}{\mathbb{R}}

\setlength{\parskip}{1em}
\setlength{\parindent}{0pt}

\usepackage{fancyhdr}
\pagestyle{fancy}
\fancyhf{}
\fancyhead[C]{\textsf{ by \href{https://youtu.be/dQw4w9WgXcQ}{Oleh\&Vanyok} }}

\begin{document}

\begin{center}

\begin{huge}
\textsf{Линейная Алгебра\\1 курс}
\end{huge}

\vspace{5mm}

\begin{LARGE}
\textsf{\textbf{Теория для экзамена 4 модуля}}
\end{LARGE}

\end{center}

\textbf{1. Дайте определение линейного функционала.\\}

\textbf{2. Дайте определение сопряженного пространства.\\}

\textbf{3. Выпишите формулу для преобразования координат ковектора при переходе к другому базису.\\}

\textbf{4. Дайте определение взаимных базисов.\\}

\textbf{5. Дайте определение биортогонального базиса.\\}

\textbf{6. Дайте определение сопряженного оператора в произвольном (не обязательно евклидовом) пространстве.\\}

\textbf{7. Сформулируйте определение алгебры над полем. Приведите два примера.\\}

\textbf{8. Сформулируйте определение тензора. Приведите два примера.\\}
Пусть есть поле $F$ и векторное пространство $V$ над этим полем, а так же $V^*$, сопряженное к $V$ и числа $p, q \in \mathbb{N} \cup \{0\}$
\\
$f: \underbrace{V \times, \dots, \times V}_{p} \times \underbrace{V^* \times, \dots, \times V^*}_{q} \to F$
\\
Называется тензором на $V$ типа $(p, q)$ и валентности $p + q$.

\textbf{9. Дайте определение эллипса как геометрического места точек. Выпишите его каноническое уравнение. Что такое эксцентриситет эллипса? В каких пределах он может меняться?\\}
Эллипсом называют геометрическое место точек, сумма расстояний от которых до двух данных точек, называемых фокусами, постоянна.
\\
Каноническое уравнение: $\frac{x^2}{a^2} + \frac{y^2}{b^2} = 1$
\\
Эксцентриситет: $\varepsilon = \frac{\sqrt{a^2 - b^2}}{a} = \sqrt{1 - \frac{a^2}{b^2}}$, $a$ - большая полуось, а $b$ - малая.
\\
Причём $\varepsilon \in [0, 1)$

\textbf{10. Дайте определение гиперболы как геометрического места точек. Выпишите её каноническое уравнение. Что такое эксцентриситет гиперболы? В каких пределах он может меняться?\\}
Гиперболой называют геометрическое место точек, модуль разности расстояний от которых до двух данных точек, называемых фокусами, постоянен.
\\
Каноническое уравнение: $\frac{x^2}{a^2} - \frac{y^2}{b^2} = 1$
\\
Эксцентриситет: $\varepsilon = \sqrt{1 + \frac{a^2}{b^2}}$
\\
Причём $\varepsilon \in (1, +\infty)$

\textbf{11. Дайте определение параболы как геометрического места точек. Выпишите её каноническое уравнение.\\}
Параболой называют геометрическое место точек плоскости, равноудаленных от данной точки (фокуса) и от данной прямой (директрисы).
\\
Каноническое уравнение: $y^2 = 2px$

\textbf{12. Сформулируйте теорему о классификации кривых второго порядка.\\}
$\forall$ кривой второго порядка $\exists$ прямоугольная декартова система координат $Oxy$, в которой уравнение этой кривой имеет один из следующих видов:
\begin{center}
    \begin{tabular}{ |c|c|c| } 
        \hline
        1&2&3
        \\
        \hline
        \text{эллипс}&\text{пустое множество}&\text{точка}
        \\
        \hline
        $\frac{x^2}{a^2} + \frac{y^2}{b^2} = 1, a \geq b > 0$&$\frac{x^2}{a^2} + \frac{y^2}{b^2} = -1$&$\frac{x^2}{a^2} + \frac{y^2}{b^2} = 0$
        \\
        \hline
    \end{tabular}
    \\
    \begin{tabular}{ |c|c| } 
        \hline
        4&5
        \\
        \hline
        \text{гипербола}&\text{пара пересекающихся прямых}
        \\
        \hline
        $\frac{x^2}{a^2} - \frac{y^2}{b^2} = 1, a > 0, b > 0$&$\frac{x^2}{a^2} - \frac{y^2}{b^2} = 0$
        \\
        \hline
    \end{tabular}
\end{center}

\textbf{13. Дайте определение цилиндрической поверхности.\\}
Рассмотрим кривую $\gamma$, лежащую в некоторой плоскости $P$, и прямую $L$, не лежащую в $P$.
\\
Цилиндрической поверхностью называют множество всех прямых, параллельных $L$ и пересекающих $\gamma$.

\textbf{14. Дайте определение линейчатой поверхности. Приведите три примера.\\}
Линейчатой называют поверхность, образованную движением прямой линии.
\\
Любой цилиндр является линейчатой поверхностью.
\\
Примеры: эллиптический цилиндр, гиперболический цилиндр, параболический цилиндр.

\textbf{15. Запишите канонические уравнения эллиптического, гиперболического и параболического цилиндров.\\}
Эллиптический цилиндр: $\frac{x^2}{a^2} + \frac{y^2}{b^2} = 1$
\\
\\
Гиперболический цилиндр: $\frac{x^2}{a^2} - \frac{y^2}{b^2} = 1$
\\
\\
Параболический цилиндр: $y^2 = 2px$

\textbf{16. Запишите канонические уравнения эллипсоида, однополостного гиперболоида, двуполостного гиперболоида.\\}
Эллипсоид: $\frac{x^2}{a^2} + \frac{y^2}{b^2} + \frac{z^2}{c^2} = 1$
\\
Однополостный гиперболоид: $\frac{x^2}{a^2} + \frac{y^2}{b^2} - \frac{z^2}{c^2} = 1$
\\
Двуполостный гиперболоид: $\frac{x^2}{a^2} + \frac{y^2}{b^2} - \frac{z^2}{c^2} = -1$

\textbf{17. Запишите канонические уравнения эллиптического параболоида, гиперболического параболоида.\\}
Эллиптический параболоид: $\frac{x^2}{a^2} + \frac{y^2}{b^2} = 2z$
\\
Гиперболический параболоид: $\frac{x^2}{a^2} - \frac{y^2}{b^2} = 2z$

\end{document}